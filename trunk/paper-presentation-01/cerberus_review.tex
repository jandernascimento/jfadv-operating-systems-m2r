\documentclass[journal]{IEEEtran}

\usepackage{color}
\usepackage{listings}
\usepackage{verbatim}
\usepackage{graphicx}
\usepackage{multicol}
\usepackage{subfig}
\usepackage{mdwlist}
\usepackage[pagebackref=true]{hyperref}%

\hyphenation{hetero-geneous op-tical net-works semi-conduc-tor}


\begin{document}
%
\title{A Case for Scaling Applications to Many-core with OS Clustering}
\author{
(reviewed by Oleg Iegorov and Jander Nascimento)}
% type author(s) between braces
\maketitle

%\section{Introduction}
This paper proposes a new way to scale operating systems for many cores
--- \emph{OS clustering}. 
%The need to scale an OS in many-core environments
%stems from the fact, that with the increasing number of cores, OS should
%efficiently share them while avoiding data contention. 
The need to scale
stems from the fact, that with the increasing number of cores in
contemporary systems, OS should
efficiently share these cores between running applications, while avoiding data contention. 
The proposed
system called \emph{Cerberus} clusters several commodity operating systems atop
a VMM and provides to applications the traditional shared memory
interface.  Cerberus adds to VMM the support for resource sharing and
communication between operating systems (running as VMs), as well as implementing system
call routing to provide an illusion for applications as running on a
single operating system.  The motivation of this approach is that
commodity operating systems (Linux, FreeBSD, etc.) can scale well with a
small number of CPU cores, while one VMM can efficiently consolidate
multiple operating systems. Thus Cerberus's approach uses several
operating systems to serve one application, without requiring any
modification of existing parallel programs.

The problem of scaling operating systems on shared memory multicore and
multiprocessor machines is not new, and there basically exist two main
approaches to solve this problem:

\begin{enumerate*}
  \item designing new operating systems from scratch;
  \item refining commodity OS kernels.
\end{enumerate*}

Cerberus uses a middle ground between these two trends, taking advantage
of both solutions' ideas.

The first main idea used by Cerberus is an extension of traditional VMMs
with support for efficient resource sharing among the clustered
operating systems. This extension is needed due to the fact that
different OS may want to share some resources (address spaces, files), as
one application can run on several OS. However, traditional VMMs (like
Xen) were designed to separate as much as possible running VMs. Solution
to this problem used by Cerberus is an addition of address range support
to VMM, as well as an efficient distributed file system.

The second idea consists in incorporation of a system call
virtualization layer. This allows processes/threads of one application
to be executed in multiple operating systems, providing users with the
illusion of running in a single OS. This layer uses the notion of
``SuperProcess'', which groups processes/threads in multiple OS to
manage the spawned processes/threads.

\section{Overview of Cerberus}

As was metioned before, one of the main contributions of Cerberus is the
modification of VMM to allow one application run on several VMs
simultaneously. This means that processes/threads belonging to one
shared-memory application now run on multiple operating systems. To
ensure system consistency in these conditions Cerberus uses the
following mechanisms.

\begin{itemize}
  \item Single Shared Memory Interface.

    Cerberus does not require modification of implemented parallel
    applications, and thus should provide to them the existing
    shared-memory interface (e.g. POSIX API). To address this issue,
    Cerberus incorporates a system-call virtualization layer which
    coordinates system calls in multiple clustered operating systems.
    All the processes/threads running on different OSes are logically
    grouped by a \emph{SuperProcess} entity.

  \item Efficient Resource Sharing.

   As traditional VMMs are designed to isolate VMs from each other, the
   other issue addressed by Cerberus was to allow an efficient sharing
   of resources (files, networking, etc.) among processes/threads. Thus,
   Cerberus implements a resource-sharing layer in both the VMM and
   operating systems.

\end{itemize}

\end{document}

